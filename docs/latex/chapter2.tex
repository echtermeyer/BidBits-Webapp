% !TEX root =  master.tex
\chapter{Normalization Analysis}
Normalization is a database design technique that reduces data redundancy and eliminates undesirable characteristics like Insertion, Update and Deletion Anomalies. Normal forms are a set of conditions or rules that a relation (table) in a database must adhere to, to qualify as a 'good' structure. This analysis will check whether the relations in the database meet at least the Third Normal Form (3NF).


\begin{table}[h]
	\centering
	\begin{tabular}{|c|p{10cm}|}
		\hline
		\textbf{Normal Form} & \textbf{Definition} \\
		\hline
		First Normal Form (1NF) & A relation is in 1NF if it contains an atomic value for each attribute (column) in a record (row). It should also have a primary key that uniquely identifies each record. \\
		\hline
		Second Normal Form (2NF) & A relation is in 2NF if it is in 1NF and all non-key attributes are fully functionally dependent on the primary key. This essentially means there is no partial dependency of any column on the primary key. \\
		\hline
		Third Normal Form (3NF) & A relation is in 3NF if it is in 2NF and no non-key attribute is transitively dependent on the primary key. \\
		\hline
	\end{tabular}
	\caption{Definitions of Normal Forms}
	\label{tab:normal_forms}
\end{table}



\textbf{1. Categorisation Table}
The Categorisation table consists of three attributes: id, category, and subcategory. The primary key is 'id', and every 'id' refers to a unique category and subcategory. There are no duplicate or redundant data, and every attribute is atomic, which satisfies 1NF.
Since there's only one candidate key (id), and all non-key attributes (category and subcategory) are fully dependent on it, 2NF is satisfied.
There's no transitive dependency as there's only one candidate key, and thus the table also satisfies 3NF.


\textbf{2. User Table}
The 'user' table has eight attributes: username, email, password, firstName, lastName, name, address, and phone. The primary key is 'username', and every 'username' refers to a unique user record.
However, the 'name' attribute is a derived attribute (it's generated always as the concatenation of 'firstName' and 'lastName'). This breaks the rule of 1NF as it introduces redundancy. To satisfy 1NF, we could remove the 'name' attribute and derive it in our queries when needed.
Assuming we consider the 'name' attribute as not violating 1NF, the table would meet the conditions for 2NF as there's only one candidate key, and all non-key attributes are fully dependent on it.
For 3NF, the table doesn't have any non-key attribute that is transitively dependent on the primary key, so it satisfies 3NF as well.


\textbf{3. Item Table}
The Item table has nine attributes: id, name, description, startingPrice, startTime, endTime, imageUrl, user\_username, and category\_id. The primary key is 'id', and every 'id' refers to a unique item.
The table is in 1NF as all attributes are atomic, and each record is unique.
The table is in 2NF as all non-key attributes are fully dependent on the primary key.
The table is in 3NF as there are no transitive dependencies.


\textbf{4. Bid Table}
The Bid table is in 1NF, 2NF, and 3NF. All attributes are atomic, each record is uniquely identified by 'id', all non-key attributes are fully dependent on the primary key, and there are no transitive dependencies.


\textbf{5. Watchlist Table}
The Watchlist table has two attributes: user\_username and item\_id. The primary key is a combination of 'user\_username' and 'item\_id', and every combination refers to a unique watchlist entry.
The table is in 1NF as all attributes are atomic, and each record is unique.
The table is in 2NF as there are no non-key attributes, so the condition of full functional dependency is trivially satisfied.
The table is also in 3NF as there are no transitive dependencies.


All tables in the given SQL code are in the Third Normal Form (3NF), assuming that the 'name' attribute in the 'user' table doesn't violate the 1NF. If it does, the 'user' table should be modified by removing the 'name' attribute to meet 1NF, 2NF, and 3NF.